\documentclass[11pt]{article}
\usepackage[margin=1in]{geometry}
\usepackage{graphicx}
\usepackage{amsmath}
\usepackage{amssymb}
\usepackage{float}
\usepackage{subcaption}
\usepackage{booktabs}

\title{Nonlinear Regression Model}

\begin{document}
\maketitle

\section{Overview}
  This week I worked on implementing the non-linear model we discussed last time. Let $g$ be whatever group we decide to calculate idleness for (Judge or County) and let $i$ be our current observation, which is a judge-county combination. The non-linear model is:

  \begin{align*}
    \mu_g \text{Days}_i &= \beta_P \text{Plea}_i + \beta_T \text{Trial}_i + \epsilon_i
  \end{align*}

  Where we interpret $\mu_g$ to be the utilization of group $g$. I was having a hard time implementing this, so I asked Walter for advice, and he suggested that I define the function $f(\mu_g,\beta_P,\beta_T) = (\mu_g \text{Days}_i - \beta_P \text{Pleas}_i - \beta_T \text{Trials}_i)^2$ and that I use a standard nonlinear optimizer to minimize it. Walter mentioned that I might have convergence problems since the model is not identified (i.e. multiplying all of the coefficients by any constants would still make the equation hold). Given that we want to interpret the $\mu_g$ as utilizations, I decided to add the constraints that $c \leq \mu_g \leq 1$ and that $\beta_T, \beta_P \geq 0$. I also hoped that this would help with identification.

\section{Results}
  The model seems very sensitive to the lower bound, $c$ that we impose. If we let $c=0$, then the optimizer sets all of the coefficients equal to 0. For other values of the lower bound (e.g. 0.6 or 0.7), the $\mu_g$ for most counties/judges ends up being set to the lower bound. I also tried $c=(0.5,0.6,0.65,0.7,0.75)$. I ran this calculating the utilizations for both judges and counties.

  \subsection{County Model}
    The results in this subsection correspond to the model: $\mu_c \text{Days}_i = \beta_P \text{Plea}_i + \beta_T \text{Trial}_i + \epsilon_i$. Where $i$ is a judge-county combination and $c$ is a county. In the table below, the numerical column names correspond to the lower bound used when computing those estimates. For example the column 0.5 contains the estimates of the nonlinear estimation with the constraint that $0.5 \leq \mu_c \leq 1$.

    \begin{table}[H]
      \centering
      % \small
      \caption{County Model}
      \begin{tabular}{lr}
\toprule
   Parameter &  Estimate \\
\midrule
   Abbeville &      0.14 \\
       Aiken &      0.23 \\
   Allendale &      0.01 \\
    Anderson &      0.23 \\
     Bamberg &      0.05 \\
    Barnwell &      0.17 \\
    Beaufort &      0.07 \\
    Berkeley &      0.16 \\
     Calhoun &      0.04 \\
  Charleston &      0.13 \\
    Cherokee &      0.22 \\
     Chester &      0.06 \\
Chesterfield &      0.16 \\
   Clarendon &      0.08 \\
    Colleton &      0.06 \\
  Darlington &      0.05 \\
      Dillon &      0.03 \\
  Dorchester &      1.00 \\
   Edgefield &      0.06 \\
   Fairfield &      0.17 \\
    Florence &      0.15 \\
  Georgetown &      0.19 \\
  Greenville &      0.19 \\
   Greenwood &      0.14 \\
     Hampton &      0.03 \\
       Horry &      0.21 \\
      Jasper &      0.14 \\
     Kershaw &      0.08 \\
   Lancaster &      0.06 \\
     Laurens &      0.11 \\
         Lee &      0.14 \\
   Lexington &      0.15 \\
      Marion &      0.11 \\
    Marlboro &      0.06 \\
   McCormick &      0.04 \\
    Newberry &      0.10 \\
      Oconee &      0.13 \\
  Orangeburg &      0.12 \\
     Pickens &      0.12 \\
    Richland &      0.16 \\
      Saluda &      0.13 \\
 Spartanburg &      0.24 \\
      Sumter &      0.14 \\
       Union &      0.13 \\
Williamsburg &      0.04 \\
        York &      0.23 \\
       BetaP &      0.02 \\
       BetaT &      1.48 \\
\bottomrule
\end{tabular}

    \end{table}

    \subsection{Judge Model}
      The results in this subsection correspond to the model: $\mu_j \text{Days}_i = \beta_P \text{Plea}_i + \beta_T \text{Trial}_i + \epsilon_i$. Where $i$ is a judge-county combination and $j$ is a judge. In the table below, the numerical column names correspond to the lower bound used when computing those estimates. For example the column 0.5 contains the estimates of the nonlinear estimation with the constraint that $0.5 \leq \mu_j \leq 1$.

      \begin{table}[H]
        \centering
        \small
        \caption{Judge Model}
        \begin{tabular}{lr}
\toprule
Parameter &  Estimate \\
\midrule
  Judge 1 &      0.13 \\
 Judge 10 &      0.18 \\
 Judge 11 &      0.14 \\
 Judge 12 &      0.15 \\
 Judge 13 &      0.13 \\
 Judge 14 &      0.16 \\
 Judge 15 &      0.14 \\
 Judge 16 &      1.00 \\
 Judge 17 &      0.16 \\
 Judge 18 &      0.12 \\
 Judge 19 &      0.21 \\
  Judge 2 &      0.17 \\
 Judge 20 &      0.13 \\
 Judge 21 &      0.10 \\
 Judge 22 &      0.26 \\
 Judge 23 &      0.11 \\
 Judge 24 &      0.19 \\
 Judge 25 &      0.26 \\
 Judge 26 &      0.22 \\
 Judge 27 &      0.17 \\
 Judge 28 &      0.17 \\
 Judge 29 &      0.16 \\
  Judge 3 &      0.17 \\
 Judge 30 &      0.09 \\
 Judge 31 &      0.13 \\
 Judge 32 &      0.19 \\
 Judge 33 &      0.26 \\
 Judge 34 &      0.21 \\
 Judge 35 &      0.16 \\
 Judge 36 &      0.13 \\
 Judge 37 &      0.11 \\
 Judge 38 &      0.21 \\
 Judge 39 &      0.27 \\
  Judge 4 &      0.12 \\
 Judge 40 &      0.16 \\
 Judge 41 &      0.11 \\
 Judge 42 &      0.32 \\
 Judge 43 &      0.16 \\
 Judge 44 &      0.28 \\
 Judge 45 &      0.08 \\
 Judge 46 &      0.45 \\
 Judge 47 &      0.17 \\
 Judge 48 &      0.16 \\
 Judge 49 &      0.10 \\
  Judge 5 &      0.31 \\
 Judge 50 &      0.19 \\
  Judge 6 &      0.25 \\
  Judge 7 &      0.19 \\
  Judge 8 &      0.18 \\
  Judge 9 &      0.22 \\
    BetaP &      0.04 \\
    BetaT &      0.33 \\
\bottomrule
\end{tabular}

      \end{table}


\end{document}
