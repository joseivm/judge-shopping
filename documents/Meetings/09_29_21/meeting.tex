\documentclass[11pt]{article}
\usepackage[margin=1in]{geometry}
\usepackage{graphicx}
\usepackage{amsmath}
\usepackage{amssymb}
\usepackage{float}
\usepackage{subcaption}
\usepackage{booktabs}

\title{Nonlinear Regression Model}

\begin{document}
\maketitle

\section{Overview}
  This week I worked on implementing the non-linear model we discussed last time. Let $g$ be whatever group we decide to calculate idleness for (Judge or County) and let $i$ be our current observation, which is a judge-county combination. The non-linear model is:

  \begin{align*}
    \mu_g \text{Days}_i &= \beta_P \text{Plea}_i + \beta_T \text{Trial}_i + \epsilon_i
  \end{align*}

  Where we interpret $\mu_g$ to be the utilization of group $g$. I was having a hard time implementing this, so I asked Walter for advice, and he suggested that I define the function $f(\mu_g,\beta_P,\beta_T) = (\mu_g \text{Days}_i - \beta_P \text{Pleas}_i - \beta_T \text{Trials}_i)^2$ and that I use a standard nonlinear optimizer to minimize it. Walter mentioned that I might have convergence problems since the model is not identified (i.e. multiplying all of the coefficients by any constants would still make the equation hold). Following your advice, I set $\mu_g=1$ for a judge/county that is not the most productive one. I just picked a one randomly, the baseline judge was Judge 41, and the baseline county was Aiken.

\section{Results}
  The model runs and yields estimates of $\mu_g$ between 0 and 1 for all judges and counties, however, the estimates of $\beta_T$ and $\beta_P$ it gives seem unreasonable. The $\beta_T$ estimates seem too low and the $\beta_P$ estimates seem too high.

  \subsection{County Model}
    The results in this subsection correspond to the model: $\mu_c \text{Days}_i = \beta_P \text{Plea}_i + \beta_T \text{Trial}_i + \epsilon_i$. Where $i$ is a judge-county combination and $c$ is a county. The county model yields $\beta_T=1.58$, which would imply that it takes a little over a day and a half to process a trial. The county model also yields $\beta_P=0.02$, which would imply judges can process 50 pleas per day.

    \begin{table}[H]
      \centering
      % \small
      \caption{County Model}
      \begin{tabular}{lr}
\toprule
   Parameter &  Estimate \\
\midrule
   Abbeville &      0.16 \\
       Aiken &      1.00 \\
   Allendale &      0.01 \\
    Anderson &      0.26 \\
     Bamberg &      0.06 \\
    Barnwell &      0.19 \\
    Beaufort &      0.08 \\
    Berkeley &      0.19 \\
     Calhoun &      0.05 \\
  Charleston &      0.16 \\
    Cherokee &      0.24 \\
     Chester &      0.07 \\
Chesterfield &      0.18 \\
   Clarendon &      0.09 \\
    Colleton &      0.07 \\
  Darlington &      0.06 \\
      Dillon &      0.04 \\
  Dorchester &      0.37 \\
   Edgefield &      0.08 \\
   Fairfield &      0.19 \\
    Florence &      0.17 \\
  Georgetown &      0.21 \\
  Greenville &      0.22 \\
   Greenwood &      0.16 \\
     Hampton &      0.04 \\
       Horry &      0.24 \\
      Jasper &      0.16 \\
     Kershaw &      0.10 \\
   Lancaster &      0.07 \\
     Laurens &      0.13 \\
         Lee &      0.15 \\
   Lexington &      0.18 \\
      Marion &      0.13 \\
    Marlboro &      0.08 \\
   McCormick &      0.05 \\
    Newberry &      0.12 \\
      Oconee &      0.15 \\
  Orangeburg &      0.15 \\
     Pickens &      0.14 \\
    Richland &      0.18 \\
      Saluda &      0.15 \\
 Spartanburg &      0.27 \\
      Sumter &      0.16 \\
       Union &      0.14 \\
Williamsburg &      0.06 \\
        York &      0.26 \\
       BetaP &      0.02 \\
       BetaT &      1.58 \\
\bottomrule
\end{tabular}

    \end{table}

    \subsection{Judge Model}
      The results in this subsection correspond to the model: $\mu_j \text{Days}_i = \beta_P \text{Plea}_i + \beta_T \text{Trial}_i + \epsilon_i$. Where $i$ is a judge-county combination and $j$ is a judge. The judge model yields $\beta_T=0.06$. The judge model also yields $\beta_P=0.01$, which would imply judges can process 100 pleas per day.

      \begin{table}[H]
        \centering
        \small
        \caption{Judge Model}
        \begin{tabular}{lr}
\toprule
Parameter &  Estimate \\
\midrule
  Judge 1 &      0.03 \\
 Judge 10 &      0.04 \\
 Judge 11 &      0.03 \\
 Judge 12 &      0.04 \\
 Judge 13 &      0.03 \\
 Judge 14 &      0.04 \\
 Judge 15 &      0.04 \\
 Judge 16 &      0.17 \\
 Judge 17 &      0.04 \\
 Judge 18 &      0.03 \\
 Judge 19 &      0.06 \\
  Judge 2 &      0.04 \\
 Judge 20 &      0.03 \\
 Judge 21 &      0.02 \\
 Judge 22 &      0.07 \\
 Judge 23 &      0.03 \\
 Judge 24 &      0.05 \\
 Judge 25 &      0.07 \\
 Judge 26 &      0.06 \\
 Judge 27 &      0.04 \\
 Judge 28 &      0.04 \\
 Judge 29 &      0.04 \\
  Judge 3 &      0.04 \\
 Judge 30 &      0.02 \\
 Judge 31 &      0.03 \\
 Judge 32 &      0.05 \\
 Judge 33 &      0.07 \\
 Judge 34 &      0.06 \\
 Judge 35 &      0.04 \\
 Judge 36 &      0.03 \\
 Judge 37 &      0.03 \\
 Judge 38 &      0.06 \\
 Judge 39 &      0.07 \\
  Judge 4 &      0.03 \\
 Judge 40 &      0.04 \\
 Judge 41 &      1.00 \\
 Judge 42 &      0.09 \\
 Judge 43 &      0.04 \\
 Judge 44 &      0.07 \\
 Judge 45 &      0.02 \\
 Judge 46 &      0.12 \\
 Judge 47 &      0.04 \\
 Judge 48 &      0.04 \\
 Judge 49 &      0.03 \\
  Judge 5 &      0.08 \\
 Judge 50 &      0.05 \\
  Judge 6 &      0.06 \\
  Judge 7 &      0.05 \\
  Judge 8 &      0.05 \\
  Judge 9 &      0.06 \\
    BetaP &      0.01 \\
    BetaT &      0.06 \\
\bottomrule
\end{tabular}

      \end{table}


\end{document}
