\documentclass[11pt]{article}
\usepackage[margin=1in]{geometry}
\usepackage{graphicx}
\usepackage{amsmath}
\usepackage{amssymb}
\usepackage{float}
\usepackage[ruled,vlined]{algorithm2e}

\title{June 3, 2021 Meeting Agenda}

\begin{document}
\maketitle

\section{Joint estimation of $\mu_p$ and $\mu_t$}
I finished implementing the ad hoc algorithm. I want to confirm that the exclusion criteria that
we are currently using is the one we want. Here are the exclusion criteria:

\begin{table}[H]
  \centering
  \caption{Plea Capacity Estimation Exclusion Criteria}
\begin{tabular}{|l|l|l|}
  \hline
\textbf{Condition}                                           & \textbf{Old Clean Day} & \textbf{New Clean Day} \\ \hline
No inconsistencies between sentencing data and calendar & Y & Y \\ \hline
Judge has at least 10 'clean' days                      & Y & Y \\ \hline
Judge has at least one sentencing event that day        & Y &   \\ \hline
Judge has no trials in this county in entire dataset    & Y &   \\ \hline
Judge is only assigned to one county that day           &   & Y \\ \hline
Judge only sentences in one county that day             &   & Y \\ \hline
Judge never has more than x sentencing events in this county &                                 & Y                               \\ \hline
Judge calendar assignment is of type "GS"               &   & Y \\ \hline
\end{tabular}
\end{table}

\begin{table}[H]
\centering
\caption{Trial Capacity Estimation Exclusion Criteria}
\label{tab:my-table}
\begin{tabular}{|l|l|l|}
  \hline
\textbf{Condition}                                           & \textbf{Old Clean Day} & \textbf{New Clean Day} \\ \hline
Judge never has more than x sentencing events in this county & Y                               & Y                               \\ \hline
Judge has at least 1 trial in this county                    & Y                               & Y \\ \hline
\end{tabular}
\end{table}

  \subsection{Problems}
    \begin{itemize}
      \item Step 4 of the ad-hoc algorithm (the part where we maximize log likelihood) is inexact. The gradient is usually not exactly equal to zero, but it does get smaller over iterations. In the last
      iteration it is at 0.002.
      \item I'm not actually using a stopping criteria for Step 4, I just run it for 100 iterations and keep the best value of $\mu_p$. I played around with the number of iterations, and the objective doesn't seem to improve after 100. 
    \end{itemize}

  \subsection{Next Steps}
  \begin{itemize}
    \item Finalize exclusion criteria.
    \item Do Poisson tail calculations.
    \item Add information about optimizer to document.
  \end{itemize}

  \subsection{Pseudocode}
  \begin{algorithm}[H]
    \SetKwInOut{Input}{input}\SetKwInOut{Output}{output}
    \SetAlgoLined
    \Input{InitialMuT, InitialMuP, tolerance}
    \Output{muP,muT}
    sdf = LoadSentencingData()\;
    pleas = GetCleanDayPleas()\;
    judgeDays = GetJudgeDays()\;
    prevMuT = 0\;
    prevMuP = 0\;
    muT = InitialMuT\;
    muP = InitialMuP\;
    \While{$(|muT - prevMuT| > tolerance)$ or $(|muP - prevMuP| > tolerance)$}{
     prevMuP = muP\;
     prevMuT = muT\;
     muP = OptimizeMuP(prevMuT,prevMuP,sdf,judgeDays,pleas)\;
     muT = EstimateMuT(muP)\;
    }
    \caption{Ad-hoc Algorithm}
    \end{algorithm}

\end{document}
