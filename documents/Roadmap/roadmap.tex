\documentclass[11pt]{article}
\usepackage[margin=1in]{geometry}
\usepackage{graphicx}
\usepackage{amsmath}
\usepackage{amssymb}

\title{Judge Shopping Roadmap}

\begin{document}
\maketitle
\tableofcontents

\section{What has been done}
  \subsection{Data}
    \subsubsection{Sentencing Data}
      \paragraph{Cleaning}
        \begin{itemize}
          \item Dropped all sentencing events with Judge 1
          \item Changed dates to be in the yyyy-mm-dd format
          \item There were two statutes that were read in as dates, '12/21/90' and '12/21/92',
          all other statutes in the format 'XX-XX-XXXX'. There were two statutes in the STATA
          data that were '12-21-2790' and '12-21-2792'. We renamed the two 'date' statutes
          based on this.
        \end{itemize}

      \paragraph{Getting expmin variable from STATA dataset}
        We noticed that the "expmin" variable in the STATA dataset was a function
        of the variables (Date,County,Circuit,Counts,Offense Seriousness,Offense Code,
        Offense Type, Sentence, Statute). In other words, sentencing events that had
        identical values for those variables had identical values of the "expmin" variable.
        So, given a sentencing event in the sentencing data, to determine its value of "expmin",
        I would find a sentencing event in the STATA data that had the same values for the
        variables mentioned above. There are 17,516 events in the STATA data with the "expmin"
        variable defined (the rest have missing values for it). Using this approach, we are able to find the "expmin" variable for
        17,516 variables in our sentencing data.
        % So, in order to add this variable to the sentencing data, I used the STATA data
        % to find the "expmin" value that corresponded to every combination of those variables
        % seen in the data.

      \paragraph{Merging with schedule data}
        We assign each Judge ID to the judge name in the schedule data whose schedule
        overlaps the most with that Judge ID. We consider the sentencing data and schedule
        data to overlap for a Judge ID and a Judge name in a specific week
        if the Judge ID in consideration has a sentencing event in a county that
        that judge name is assigned to in that week. So, if in Week 12 judge 'John Doe'
        is assigned to 'Aitken' county, and we see a sentencing event for Judge ID $n$
        in 'Aitken' county on Week 12, we would consider their schedules to be overlapping in
        that week. Note, this procedure yields a mapping in which Judge ID's are assigned to
        Judges based on alphabetical order.

    \subsubsection{Schedule Data}
      \paragraph{Cleaning}
        \begin{itemize}
          \item We renamed Judge "COOPER" to "COOPER TW"
          \item We removed non-alphabetic characters from judge names
          \item We removed leading and trailing whitespace from judge names
          \item We parsed the judge assignments where multiple counties are listed
          and created a separate row in the data for each assignment. So, for example,
          if Judge Cooper was assigned to Williamsburg and Aitken county on Week 12,
          There would be one row showing only the Williamsburg assignment and another row
          showing only the Aitken assignment, both with the same week date.
        \end{itemize}

  \subsection{Parameter Estimation}
    \subsubsection{Conviction Probability at Trial - $\theta$}
      We currently estimate this using logistic regression. We currently are using the
      following variables to predict this: Black, Offense Type, Offense Seriousness. \textbf{Note:} We are only using the defendants that went to trial to estimate this, there are only ~270 defendants that went to trial in the data.

    \subsubsection{Expected Sentence Length if Convicted - $\tau$}
      We currently estimate this using Negative Binomial Regression. We use the Cameron-Trivedi
      test for overdispersion to choose the overdispersion parameter. \textbf{Note:} We are only using the defendants that went to trial to estimate this, there are only ~270 defendants that went to trial in the data.

    \subsubsection{Defendant Cost of Trial - $c_d$}
      We are currently estimating this using only the first method described in Nasser's document. In this method, we use the subset of cases where the sentence, $s$ is less than $u_j(\theta \tau)$. In these cases,
      our model implies that $c_d = s - \theta \tau$.

    \subsubsection{Judge maximum and minimum plea - $l_j(\theta \tau),u_j(\theta \tau)$}
      We currently estimate this using a K-nearest neighbors approach. Given a defendant's
      $\theta \tau$, we find the K defendants in the judge's past pleas with the
      most similar values of $\theta \tau$. We then pick the minimum and maximum of these pleas
      to determine $l_j(\theta \tau),u_j(\theta \tau)$.

    \subsubsection{County Arrival Rates - $\lambda_c$}
      We set each county's arrival rate to be equal to the average number of sentencing events per week in that county, as observed in the data.

    \subsubsection{Service Rates - $\mu_p, \mu_t$}
      Service rates are currently being calculated as described in Nasser's document.

  \subsection{Simulation}
    \begin{itemize}
      \item We first assign each county one judge for each time period. The time periods
      here are discrete and we think of them as weeks. There are 50 judges and 46 counties,
      in each week, we randomly draw 46 judges (without replacement) and assign them to a county in the order they are drawn. Counties are sorted alphabetically. So the first judge to be drawn is assigned to County A, the second to County B, and so on. Judges capacity is determined by our estimation of the plea service rate.
      \item In each time period, $t$, we iterate over the different counties. We simulate defendant arrivals for a given county, $c$, as follows: first, we determine the number of arrivals, $n_{ct}$ by drawing from a Poisson distribution with mean $lambda_c$. We then draw $n_{ct}$ defendants from county
      $c$'s past defendants.
      \item Each defendant then chooses from the available judges that will be in
      county $c$ in the next $r$ weeks. The defendant chooses the judge, $j$ that minimizes his expected cost, $\min(\theta \tau + c_d,u_j(\theta \tau)) + k(j)d$, where $c_d$ is the defendant's cost of going to trial, $k(j)$ is the
      number of time periods until judge $j$ will be in county $c$, and $d$ is the cost of delay.
      \item Once a defendant chooses a judge, we reduce that judge's capacity for the week in which he will sentence that defendant.
    \end{itemize}

  \subsection{Analysis}
    We currently only compute defendant backlog, and mean and variance.

\section{What remains to be done}
  \subsection{Data}
    \subsubsection{Schedule Data}
      \begin{itemize}
        \item Parse the days from the schedule data, so we know the exact dates judges were assigned to counties
        \item Parse the assignment types
      \end{itemize}

  \subsection{Parameter Estimation}
    \subsubsection{Conviction Probability at Trial - $\theta$}
      \begin{itemize}
        \item Look into Hurdle model used by Hester.
        \item Use more covariates to predict this.
      \end{itemize}

    \subsubsection{Expected Sentence Length if Convicted - $\tau$}
      \begin{itemize}
        \item Look into Hurdle model used by Hester.
        \item Use more covariates to predict this.
      \end{itemize}

    \subsubsection{Defendant Cost of Trial - $c_d$}
      \begin{itemize}
        \item Figure out how to implement the MLE approach described in Nasser's document
        that uses information from sentencing events other than those in which $s < u_j$
      \end{itemize}

    \subsubsection{Judge maximum and minimum plea - $l_j(\theta \tau),u_j(\theta \tau)$}
      \begin{itemize}
        \item Look into convex hull approach.
        \item Remove each defendant from K nearest neighbors.
      \end{itemize}

    \subsubsection{Service Rates - $\mu_p, \mu_t$}
      \begin{itemize}
        \item Study documentation to refine exclusion criteria for "clean days".
        \item Try approach 1 for estimating plea service rate. Figure out how to enumerate the tuples of interest.
        \item Try approach 2 for estimating plea service rate.
        \item Try approach 3 for estimating plea service rate.
        \item Use updated plea service rate estimate to re-estimate trial service rate.
      \end{itemize}

  \subsection{Simulation}
    \begin{itemize}
      \item Confirm that current way of simulating arrivals works.
      \item Fix the processing of backlogged defendants.
      \item Consider only assigning judges to the counties they are assigned to in the data. Or taking into consideration geography when assigning judges.
      \item Incorporate defendants' decision to go to trial.
      \item Incorporate capacity reductions for judges from time spent on trials.
      \item Consider assigning more than one judge per week for busier counties.
    \end{itemize}

  \subsection{Analysis}
    \begin{itemize}
      \item Think about what other metrics to compute.
      \item Consider analyzing county level outcomes (eg intra and inter county variance).
    \end{itemize}

\end{document}
