\documentclass[11pt]{article}
\usepackage[margin=1in]{geometry}
\usepackage{graphicx}
\usepackage{amsmath}
\usepackage{amssymb}

\title{Judge Shopping To Do List}

\begin{document}
\maketitle
\tableofcontents

\section{To Do List}
  \subsection*{Current Priorities}
    \begin{itemize}
      \item Replicate Nasser's discrepancy analysis between the sentencing data and
      the schedule data. This includes: 1. reading the section in the document detailing the
      particularities of the schedule data 2. Running some tests to make sure  daily schedule data
      was parsed correctly.
      \item Read background material and relevant sections in the document to refine or
      confirm the criteria for "clean days".
      \item Implement "clean days" exclusion and service rate estimation procedures.
    \end{itemize}

  \subsection{Data}
    \subsubsection{Schedule Data}
      \begin{itemize}
        \item Parse the assignment types (e.g. CP, GS, CC)
      \end{itemize}

  \subsection{Parameter Estimation}
    \subsubsection{Conviction Probability at Trial - $\theta$}
      \begin{itemize}
        \item Look into Hurdle model used by Hester.
        \item Use more covariates to predict this.
      \end{itemize}

    \subsubsection{Expected Sentence Length if Convicted - $\tau$}
      \begin{itemize}
        \item Look into Hurdle model used by Hester.
        \item Use more covariates to predict this.
      \end{itemize}

    \subsubsection{Defendant Cost of Trial - $c_d$}
      \begin{itemize}
        \item Figure out how to implement the MLE approach described in Nasser's document
        that uses information from sentencing events other than those in which $s < u_j$
      \end{itemize}

    \subsubsection{Judge maximum and minimum plea - $l_j(\theta \tau),u_j(\theta \tau)$}
      \begin{itemize}
        \item Look into convex hull approach.
        \item Remove each defendant from K nearest neighbors.
      \end{itemize}

    \subsubsection{Service Rates - $\mu_p, \mu_t$}
      \begin{itemize}
        \item Study documentation to refine exclusion criteria for "clean days".
        \item Try approach 1 for estimating plea service rate. Figure out how to enumerate the tuples of interest.
        \item Try approach 2 for estimating plea service rate.
        \item Try approach 3 for estimating plea service rate.
        \item Use updated plea service rate estimate to re-estimate trial service rate.
      \end{itemize}

  \subsection{Simulation}
    \begin{itemize}
      \item Confirm that current way of simulating arrivals works. Current way is to draw the number of arrivals for each period from a Poisson($\lambda_c$) for each county, $c$, and then draw that number of defendants from the county's past defendants, with replacement.
      \item Fix the processing of backlogged defendants.
      \item Consider only assigning judges to the counties they are assigned to in the data. Or taking into consideration geography when assigning judges.
      \item Incorporate defendants' decision to go to trial.
      \item Incorporate capacity reductions for judges from time spent on trials.
      \item Consider assigning more than one judge per week for busier counties.
    \end{itemize}

  \subsection{Analysis}
    \begin{itemize}
      \item Think about what other metrics to compute.
      \item Consider analyzing county level outcomes (eg intra and inter county variance).
    \end{itemize}

\end{document}
